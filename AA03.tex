\documentclass[12pt]{scrartcl}

% packages
\usepackage[
    a4paper, total={18cm, 26cm},
    left=0.75in,
    right=0.75in,
    top=0.75in,
    bottom=0.50in,
    footskip=15pt
]{geometry}
\usepackage{lastpage}
\usepackage{graphicx} % \includegraphics
\usepackage{amsmath} % math
\usepackage{steinmetz} % \phase
\usepackage{import} % \import
\usepackage{esdiff} % \diff
\usepackage{fancyhdr} % header and footer

% configs
\setlength{\parindent}{0pt}
\graphicspath{ {./PrintsQuestoes/} }

% comandos
\renewcommand{\familydefault}{\sfdefault}
\newcommand{\un}[1]{\;\textrm{#1}}
\newcommand{\logo}{\quad \Rightarrow \quad}
\newcommand{\fase}[1]{\ensuremath{\phase{{#1}^{\circ}}}}

\begin{document}

\pagestyle{fancy}

\fancyhead{}
\fancyhead[R]{Matrícula: 2020028101}
\fancyhead[L]{Eletromagnetismo Computacional - Atividade Avaliativa 03}
\fancyfoot{}
\fancyfoot[R]{Pág. \thepage \; / \pageref{LastPage}}

\begin{center}
    Aluno: Raphael Henrique Braga Leivas \\
    Matrícula: 2020028101 \\[20pt]
    Código fonte LaTeX desse arquivo pode ser visto em meu GitHub pessoal: \\
    https://github.com/RaphaelLeivas/eletro-comp-2023-1
\end{center}

\hrule
\vspace{1cm}

Precisamos resolver o problema unidimensional para $f = \rho = 5$, $q = h = 1$.

\begin{equation}\label{eqProblema}
    \begin{cases}
        u_{xx} + f = 0 \\
        \noalign{\vskip9pt}
        u(1) = q       \\
        \noalign{\vskip9pt}
        -u_x(0) = h
    \end{cases}
    \logo
    \begin{cases}
        u_{xx} = - \rho \\
        \noalign{\vskip9pt}
        u(1) = q        \\
        \noalign{\vskip9pt}
        -u_x(0) = h
    \end{cases}
\end{equation}

Em \eqref{eqProblema}, temos uma EDO de segunda ordem com duas
condições de contorno.

\subsection*{1 Solução Analítica}

Para $\rho$ constante, podemos resolver o problema por integração direta.
Integrando ambos lados da EDO de \eqref{eqProblema}, temos

\[
    u_x = \int -\rho \, dx
    \logo
    u_x = - \rho x + C
\]

Usando a segunda condição de contorno,

\[
    -u_x(0) = h
    \logo
    -u_x(0) = + \rho (0) - C
    \logo
    h = + \rho (0) - C
    \logo
    C = - h
\]

Assim, temos a expressão de $u_x$

\begin{equation}\label{eqUx}
    u_x = - \rho x  - h
\end{equation}

Novamente integramos \eqref{eqUx} em ambos lados com respeito a $x$

\[
    u = \int -\rho x - h \, dx
    \logo
    u = - \rho \frac{x^2}{2} - hx + C
\]

Usando a primeira condição de contorno,

\[
    u(1) = q
    \logo
    u(1) = - \rho \frac{1^2}{2} - h(1) + C
    \logo
    q = - \frac{\rho}{2} - h + C
    \logo
    C = \frac{\rho}{2} + h + C
\]

Assim, temos a solução analítica de \eqref{eqProblema}

\[
    u = - \rho \frac{x^2}{2} - hx + \frac{\rho}{2} + h + C
\]

\begin{equation}\label{eqU}
    u = \frac{\rho}{2} \left(1 - x^2\right) + h\left(1 - x\right) + q
\end{equation}

Substituindo os dados do enunciado, temos

\[
    u = \frac{5}{2} - \frac{5x^2}{2} + 1 - x + 1
\]

\begin{equation}\label{eqUsubstituida}
    u(x) = -\frac{5}{2}x^2 - x + \frac{9}{2}
\end{equation}

\subsection*{2 Introdução FEM}

É possível também encontrar a solução de \eqref{eqProblema} pelo método de elementos finitos (FEM). Antes disso, vamos descrever um passo a passo
reproduzível para aplicar o FEM para resolver EDOs com condições de contorno da forma de \eqref{eqProblema}. \\

Pelo FEM, temos que a solução de \eqref{eqProblema} é dada por

\begin{equation}\label{eqUhFEM}
    u^h(x) = \sum_{A=1}^n d_AN_A + qN_{n+1}
\end{equation}

onde

\begin{itemize}
    \item $n$: graus de liberdade escolhido para a solução;
    \item $d_A$: A-ésimo coeficiente que se deseja encontrar;
    \item $N_A$: A-ésima função linear de forma;
    \item $qN_{n+1}$: Termo para garantir condições de contorno de $u^h(x)$.
\end{itemize}

Note que as funções $N_A(x)$ podem ser arbitrariamente escolhidas, desde que sejam lineares e satisfaçam a condição de $N_A(1) = 0$. 
Contudo, usamos um processo para determinar as funções $N_A(x)$ de modo a facilitar o método:

\end{document}